% You insert your abstract in the space below.

This thesis develops a new iOS app named EyesOnCrops to visualize NASA's Normalized Difference Vegetation Index (NDVI) data in a user-friendly way. NDVI data, in the range between -1 and 1, represent the Earth’s surface vegetation by measuring the difference between near-infrared (for which vegetation strongly reflects) and red light (for which vegetation absorbs). A large positive NDVI value means good vegetation and hence is colored green. NDVI is a huge dataset of over 10 TB, with spatial resolution in ecodistrict and time resolution in 8 days. EyesOnCrops is the first technology that can conveniently and instantly deliver the NDVI data in both color maps and digital data. A user can use an iPhone, an iPOD, or an iPad to visualize the NDVI data at the spatial resolution levels of countries, states, or ecodistricts. Our tool will be extremely valuable to farmers, hedge fund traders and insurance business in addition to teachers and scientists. The thesis includes both the development theory and a user manual of EyesOnCrops. The first chapter reviews the available smartphones technology for spatial data visualization and gives brief introduction about app development methods. The second chapter describes NDVI applications and their significance and explains the EyesOnCrops user manual. EyesOnCrops has an important function of exporting data in the CSV format via email. The third chapter describes the toolkit - NDVI Dataset and its structure, Database Schema, Wireframes of the application and other software packages used for the creation of EyesOnCrops. The fourth chapter describes the process of using an app development toolkit to get NDVI data and converting the data into the format customizable for our mobile application. The fifth chapter made some statistical analysis of the NDVI data. In the end, this thesis also explains how the EyesOnCrops technology can be portable to the applications in other fields.


