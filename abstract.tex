% You insert your abstract in the space below.


This research addresses current trends in big data, and how to apply them to mathematical, statistical, and climate science applications.

The first chapter review several big data visualization tools currently available. This section describes the utility of such tools, and draws questions on how to attract more users.

The second chapter describes a  toolkit for snow-cover area calculation and display (SACD) based on the Interactive Multisensor Snow and Ice Mapping System (IMS). The paper uses the Tibetan Plateau region as an example to describe the toolkit's method, results, and usage. 
The National Snow and Ice Data Center (NSIDC) provides to the public IMS, a well-used system for monitoring the snow and ice cover. The newly developed toolkit is based on a simple shoe-lace formula for a grid box area on a sphere and can be conveniently used to calculate the total area of snow cover given the IMS data. The toolkit has been made available as an open source Python software on GitHub. The toolkit generates the time-series of the daily snow-covered area for any region over the Northern Hemisphere from 4 February 1997. The toolkit also creates maps showing snow and ice coverage with an elevation background. The Tibetan Plateau (TP) region $(25^{\circ}-45^{\circ}N) \times (65^{\circ}-105^{\circ}E)$ is used as an example to demonstrate our work on SACD. The IMS products at 24, 4, and 1 km resolutions include each grid's latitude and longitude coordinates that are used to calculate the grid box's area using the shoe-lace formula. The total TP area calculated by the sum of the areas of all the grid boxes approximates the true spherical TP area bounded by $(25^{\circ}-45^{\circ}N) \times (65^{\circ}-105^{\circ}E)$ with a difference  0.046 \% for the 24 km grid and 0.033 \% for the 4 km grid.  The differences in the snow-cover area reported by the 24 km and 4 km grids vary between -2.34 and 6.24 \%. The temporal variations of the daily TP snow cover are displayed in time series from 4 February 1997 to present with 4 km and 24 km resolutions. 

The third chapter extends toolkit into a fully fledged web app for the Argo dataset. The Argo program has generated close to two million temperature, salinity, and pressure (T/S/P) profiles in the upper 2000 meters of the ocean. A new web app named Argovis at \url{www.argovis.com} provides easy access to Argo profile data and gridded products for both scientists and the general public. The RESTful application allows users or even other apps to interact with a database through the URL. In this case, users query a MongoDB database filled with Argo profiles. Users input a set of latitude-longitude coordinates, date range, and pressure range. The Argovis app sends a response: either raw JSON or HTML page with profile measurements and metadata that fall within these queries. Jupyter notebooks written in Python outline the API interface. Time series and gridded products are generated by API, offering a way for researchers to tailor the web app to their specific needs. 
The topics covered in this thesis entail big data sets and how to visualize and retrieve data quickly. The issues posed in this thesis extend past 3D/4D climate data sets. Open source toolkits and web apps for data visualization and accessibility pertain big data sets in general. 


