% You insert your abstract in the space below.

This research addresses current trends in big data and how to apply them to climate science applications so as to reach to common people through a mobile application.
EyesOnCrops is an native iOS application developed for visualization of NASA's NDVI Dataset in a very user friendly manner.
Normalized Difference Vegetation Index (NDVI) computes vegetation by measuring the difference between near-infrared (for which vegetation strongly reflects) and red light (for which vegetation absorbs). NDVI always ranges from -1 to +1. More positive the value means more the green area i.e better vegetation index.
The first chapter focuses on reviewing the spatial data visualization smartphones applications which are available and a brief introduction about app development methods.
The second chapter tells about application and its significance, Also explains the user manual of the application EyesOnCrops.
The third chapter describes the toolkit - NDVI Dataset and its structure, Database schema, Wireframes of the application and other softwares used for creation of the application.
The fourth chapter uses the toolkit to describe the process of getting NDVI data, processing it so as to make it customisable for the mobile application. This chapter also tells us about the development story of the app with screens designed in it as well.
The fifth chapter depicts the analytics side of application part of the mobile app and explains how it can be used as an analysis tool in various fields.

The thesis also involve big data sets and how to visualize and retrieve data quickly. The issues posed in this thesis extend past 3D/4D climate data sets. Open source tool kits and mobile apps for data visualization and accessibility pertain big data sets in general. 


