\chapter{DEVELOPMENT OF THE iOS APP}
\label{chap:development of the app}

The development of the iOS app is divided into two areas which are :

\begin{itemize}
    \item Front-end
        \begin{itemize}
            \item App designing
            \item Giving those screens the programming logic
        \end{itemize}

    \item Back-end
        \begin{itemize}
            \item Web services required for \gls{json} parsing between database and the front-end
    \end{itemize}
\end{itemize}

\section{Process of getting data from NASA's server}
\section{Mobile App screens and their significance}

\subsection{Sign-up via Email}
\subsection{Login process}
\subsection{Home screen}
\subsection{Sliding Menu}
\subsection{Filter process and its sub-screens}

\section{API usage}
\section{Softwares used in making of the app}

Visualization Apps like Argovis have served two purposes. The first purpose is to serve the community; the second is to expand the community. Global data sets in the oceanographic and earth science disciplines could potentially benefit from such apps, with minimal cost to their endeavors. The architecture of Argovis is designed for high traffic demands at low computation by placing visualization loads on the client side. The server side acts as a database and web server only; this allows a relatively high traffic web app be hosted and maintained at a low cost. Moreover, the software used is open source. 

Governmental agencies NOAA and NASA are required to release their data. The amount of data gathered and released is now on the petabyte scale. Without accessibility, this amounts to little more than an archive, accessible only to domain experts. This thesis proposes that agencies and groups design their data interfaces with a user interface in mind. Argovis is designed around the following data acquisition procedure: 
\begin{enumerate}
  \item browse datasets. 
  \item assess its relevance to their needs.
  \item visualize/select their region of interest.
  \item download data in a format they can use.
\end{enumerate}
An additional consideration would be to include a discussion forum to the web app, similar to \url{www.kaggle.com} \cite{kaggle} competitions, where each dataset has its discussion forum, where users post their code, figures and results in \gls{Jupyter} notebooks. Forums allow outside users to contribute to the project. As the institutions themselves may be limited in manpower and funding, they may rely on forums for additional code, tutorials or discussion. For example, Argovis has an API in Python, but users proficient in other scientific languages can write APIs in other languages, such as R, Matlab, and Julia. A forum section of the site works as a source of feedback for the development team. Questions, requests, complaints can be posted and be resolved/answered by either other users or the developers themselves.

The scientific community can become bogged down in data storage problem. Open, community-driven web applications to big data visualization and analysis can help users float on this sea of data.