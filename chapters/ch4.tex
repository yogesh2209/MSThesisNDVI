\chapter{CONCLUSIONS AND DISCUSSION}
\label{chap:conclusion}

Big data does not need to be viewed all at once. Humans do not need process petabytes at a time. A more useful application to data visualization is to have the data be accessible through an easy to use interface. This technology exists in the form of videos, where users view one frame at a time, or a browser, able to navigate complex website networks. In the case of scientific research, data should be easily divided into regions of interest. Scientists interested in the Tibetian Plateau do not need to download the Entire IMS dataset.

The tools discussed in this thesis is by its nature supplementary to their respective communities. IMS users can now make both local and global snow and ice area assertions with the calculations provided by TSM. It provides the gridded satellite community a customizable toolkit that can be changed to suit individual requirements. The source code made available at Github \cite{git_proj} and is explained in \url{www.itsonlyamodel.us} \cite{tibet_snow_man} aids a researcher to use TSM for their own needs. They may wish to project using another geo-coordinate system or area projection for example. Users can extend the TSM to another project that uses another gridded data set whose cell areas are unknown. Area inferencing such as TSM's snow/ice cover is extensible to vegetation, cloud cover, ocean color, et cetera. As remote sensing technology and algorithms used by the fleet of satellites products improve, more of this sort of inferencing will become more common. Consequentially, climate inferencing based on satellite methods puts pressure on the IMS community to include uncertainty in their measurements. The IMS is a blend of satellite sensing products \cite{nat_ice}, with their resolution and uncertainty. The addition of atmospheric scattering, cloud cover, vegetation, and elevation effects further complicate the physics \cite{basist}. Reducing the accuracy of these products is nontrivial, but with the increase of area-based inferencing, is predicted to be more in demand. TSM's visualization section places additional pressure, as it becomes accessible to more users by the ease of selection and time series visualization. Future work planned on TSM involves creating a 3D (latitude-longitude-time) visualization app that professional and public can use, in a similar app to \url{www.argovis.com} as discussed in Chapter~\ref{chap:argo}.

Visualization Apps like Argovis have served two purposes. The first purpose is to serve the community; the second is to expand the community. Global data sets in the oceanographic and earth science disciplines could potentially benefit from such apps, with minimal cost to their endeavors. The architecture of Argovis is designed for high traffic demands at low computation by placing visualization loads on the client side. The server side acts as a database and web server only; this allows a relatively high traffic web app be hosted and maintained at a low cost. Moreover, the software used is open source. 

Governmental agencies NOAA and NASA are required to release their data. The amount of data gathered and released is now on the petabyte scale. Without accessibility, this amounts to little more than an archive, accessible only to domain experts. This thesis proposes that agencies and groups design their data interfaces with a user interface in mind. Argovis is designed around the following data acquisition procedure: 
\begin{enumerate}
  \item browse datasets. 
  \item assess its relevance to their needs.
  \item visualize/select their region of interest.
  \item download data in a format they can use.
\end{enumerate}
An additional consideration would be to include a discussion forum to the web app, similar to \url{www.kaggle.com} \cite{kaggle} competitions, where each dataset has its discussion forum, where users post their code, figures and results in \gls{Jupyter} notebooks. Forums allow outside users to contribute to the project. As the institutions themselves may be limited in manpower and funding, they may rely on forums for additional code, tutorials or discussion. For example, Argovis has an API in Python, but users proficient in other scientific languages can write APIs in other languages, such as R, Matlab, and Julia. A forum section of the site works as a source of feedback for the development team. Questions, requests, complaints can be posted and be resolved/answered by either other users or the developers themselves.

The scientific community can become bogged down in data storage problem. Open, community-driven web applications to big data visualization and analysis can help users float on this sea of data.