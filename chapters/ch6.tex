\chapter{CONCLUSIONS AND DISCUSSION}
\label{chap:conclusion}

\section{Conclusion}

The main issue addressed in this thesis is the issue of big data visualization in a mobile application with a focus on user friendliness interface. The tools in this app are designed to serve many communities. Currently, the mobile application, EyesOnCrops, supports Swift 4.1 language and Xcode 9.2. EyesOnCrops steps for data acquisition method:
\begin{enumerate}
  \item open the app and log in. 
  \item select year and date or any filter options relevance to their needs.
  \item visualize/select their region of interest.
  \item download data so that they can use.
\end{enumerate}

At this time users can make use of these features to visualize the NDVI / Anomaly mean data with the analysis methods explained in chapter 5. 
This chapter concludes with recommendations for major two
categories of further improvement: Data generic mobile app and analysis inside the app.

\section{Future Scope}

An extra thought is to incorporate a discussion forum on the mobile application like a open chat box, similar to \url{https://www.crowdanalytix.com/} rivalries, where each dataset has its dialog discussion with concentration in Artificial Intelligence, where clients post their code, figures and results in note pads. These permit outside clients to be part of the discussions. Some startup companies look for resources, code, instructional exercises or discourse. For instance, this paper introduced NDVI Data API in Python, yet clients capable in other programming languages can compose APIs in different languages, like, R, Perl, MATLAB, Java or PHP. Questions, can be asked, objections can be posted and be settled/replied by either different clients or the engineers themselves. \\

Secondly, other future aspect would be to make this app completely generic by letting people pick the dataset they want to visualize. For example, there should be a dropdown menu of datasets coming from servers, filters, exporting data formats, and databases. Architecture should be generic as well so that it can hold any dataset and let users visualize their data in the app. \\

Lastly, another enhancement which can be done in future is to have analysis mechanism in the app. For example, if user wants to analyze a specific country for last 10 years, there should be a mechanism to see the mean, standard deviation on the graph in the app itself. 
