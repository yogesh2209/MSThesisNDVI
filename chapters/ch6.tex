\chapter{CONCLUSIONS AND DISCUSSION}
\label{chap:conclusion}

\section{CONCLUSION}

The tools discussed in this thesis is by its nature supplementary to their respective communities. Users can now make use of these features to visualize the NDVI / Anomaly mean data with the analysis methods provided explained in chapter 5. 

EyesOnCrops is composed around the accompanying data acquisition method:
\begin{enumerate}
  \item open the app and log in. 
  \item select year and date or any filter options relevance to their needs.
  \item visualize/select their region of interest.
  \item download data so that they can use.
\end{enumerate}


\section{FUTURE SCOPE}


An additional consideration would be to include a discussion forum to the web app, similar to \url{www.kaggle.com} \cite{kaggle} competitions, where each dataset has its discussion forum, where users post their code, figures and results in notebooks. Forums allow outside users to contribute to the project. As the institutions themselves may be limited in manpower and funding, they may rely on forums for additional code, tutorials or discussion. For example, Argovis has an API in Python, but users proficient in other scientific languages can write APIs in other languages, such as R, Matlab, and Julia. A forum section of the site works as a source of feedback for the development team. Questions, requests, complaints can be posted and be resolved/answered by either other users or the developers themselves.

The scientific community can become bogged down in data storage problem. Open, community-driven web applications to big data visualization and analysis can help users float on this sea of data.