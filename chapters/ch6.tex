\chapter{CONCLUSIONS AND DISCUSSION}
\label{chap:conclusion}

\section{Conclusion}

The focal issue addressed in this theory is the issue of big data visualization in a mobile application with a focus on user friendliness in terms of user interface. The tools discussed in this thesis is by its nature supplementary to their respective communities. Currently, the mobile application, EyesOnCrops, supports Swift 4.1 language and Xcode 9.2. 

EyesOnCrops is composed around the accompanying data acquisition method:
\begin{enumerate}
  \item open the app and log in. 
  \item select year and date or any filter options relevance to their needs.
  \item visualize/select their region of interest.
  \item download data so that they can use.
\end{enumerate}

Users can now make use of these features to visualize the NDVI / Anomaly mean data with the analysis methods explained in chapter 5. 
This chapter concludes with recommendations for major two
categories of further improvement: Data generic mobile app and analysis inside the app.

\section{Future scope}

An extra thought is incorporate a discussion forum on the mobile application like a open chat box, like \url{https://www.crowdanalytix.com/} rivalries, where each dataset has its dialog discussion with concentrating on Artificial Intelligence, where clients post their code, figures and results in note pads. Discussions permit outside clients to add to the task. As the establishments themselves might be restricted in labor and financing, they may depend on gatherings for extra code, instructional exercises or discourse. For instance, we have made NDVI Data API in Python, yet clients capable in other programming languages can compose APIs in different languages, for example, R, Perl, Matlab, Java or PHP. A discussion area of the site functions as a wellspring of criticism for the advancement group. Questions, asks for, objections can be posted and be settled/replied by either different clients or the engineers themselves. \\

Secondly, one other future aspect would be to make this app completely generic by letting people pick the dataset they want to visualize. For example, there should be a drop down menu of datasets coming from server, filters, exporting data formats, and database architecture should be generic too so that it can hold any dataset and ler users visualize their data in the app. \\

Last but not the least, another enhancement which can be done in future is to have analysis mechanism in the app. For example, if user wants to analyze a specific country for last 10 years, there should be a mechanism to see the mean, standard deviation on the graph in the app itself. 
