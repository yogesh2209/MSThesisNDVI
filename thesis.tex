 
% Example SDSU Mathematics LaTeX Thesis.
% Lines beginning with % are comments and are ignored.
% 
% The class file sdsu-thesis.cls must be in the current directory or
% installed with the other classes as per standard LaTeX installation.
% 
% To generate run these commands:
% latex  thesis
% bibtex thesis
% latex  thesis
% latex  thesis
% Then you need to use the dvips command to get postscript output
% 
% See the README file for more information
% 
\documentclass{sdsu-thesis}
% 
% For early printouts to save paper use the savepaper option as
% 
%\documentclass[savepaper]{sdsu-thesis}
% 
% 
% This will make things single spaced, use small font and smaller
% margins.  Stuff will be formatted differently if you don't use this
% option but it's useful to basically see (read) what you typed so far
% on paper without wasting much paper.  You might want to also comment
% out the front matter and backmatter if printing out in savepaper
% mode to save paper there.  Do not use this option on your final
% printout as it doesn't satisfy the thesis manual requirements.

% Also if you want to use double spacing rather then singlespacing (if
% your thesis is very short, say 25 pages or less), then use the
% `doublespace' option as
% 
% \documentclass[doublespace]{sdsu-thesis}

% For including graphics use
% (Info) http://en.wikibooks.org/wiki/LaTeX/Importing_Graphics
%
% NOTE: My *may* need the graphicx package to get the correct
% page-size (letter) for your document... Some environments,
% e.g. TeXnicCenter default to 'a4' page size.
\usepackage{epsfig}

%Tyler Tucker's included packages
\usepackage{color,soul}
\usepackage{verbatim}
\usepackage{cancel}
\usepackage{float}
\usepackage{graphicx}
\usepackage[numbers]{natbib}
\usepackage{algorithm}
\usepackage[noend]{algpseudocode}

\usepackage{listings}
\usepackage{color}
\lstset{
basicstyle=\small\ttfamily,
columns=flexible,
breaklines=true
}
\graphicspath{ {Figures/}{Figures/ch1/}{Figures/ch2/}{Figures/ch3/} }
\usepackage{hyperref}
\hypersetup{
    colorlinks=true,
    linkcolor=black,
    filecolor=magenta,      
    urlcolor=black,
}

\definecolor{dkgreen}{rgb}{0,0.6,0}
\definecolor{gray}{rgb}{0.5,0.5,0.5}
\definecolor{mauve}{rgb}{0.58,0,0.82}
\lstset{frame=tb,
  language=Python,
  aboveskip=3mm,
  belowskip=3mm,
  showstringspaces=false,
  columns=flexible,
  basicstyle={\small\ttfamily},
  numbers=none,
  numberstyle=\tiny\color{gray},
  keywordstyle=\color{blue},
  commentstyle=\color{dkgreen},
  stringstyle=\color{mauve},
  breaklines=true,
  breakatwhitespace=true,
  tabsize=3
}
\lstdefinelanguage{JavaScript}{
  keywords={typeof, new, true, false, catch, function, return, null, catch, switch, var, if, in, while, do, else, case, break},
  keywordstyle=\color{blue}\bfseries,
  ndkeywords={class, export, boolean, throw, implements, import, this},
  ndkeywordstyle=\color{darkgray}\bfseries,
  identifierstyle=\color{black},
  sensitive=false,
  comment=[l]{//},
  morecomment=[s]{/*}{*/},
  commentstyle=\color{purple}\ttfamily,
  stringstyle=\color{red}\ttfamily,
  morestring=[b]',
  morestring=[b]"
}


%\usepackage{glossaries}
% These packages may also be useful for pictures...
% \usepackage{color}
% \usepackage{eepic}
% \usepackage{epic}
% \usepackage{grapic}


% Since this is a math thesis, you quite likely want these:
\usepackage{amsmath}
\usepackage{amsfonts}
\usepackage{amssymb}
\usepackage{amsthm}
%\usepackage[linesnumbered,ruled]{algorithm2e}
%\usepackage{algorithmic}

% This makes captions *bold*
%
%(NOTE): Adding "justification=justified,singlelinecheck=false" to the
%        list of options will force all captions (including "short
%        one-line") to be left justified.  This is technically
%        required by the DTM, but looks ugly.
\usepackage[bf,labelsep=period,textfont=bf]{caption}

% Other useful packages for theses (see LaTeX docs for descriptions of these)
% 
% For the \vref commands that also prints out the reference page
% \usepackage{varioref}
% 
% For including computer code
% \usepackage{alltt}
% 
% For the \url{http://foo.com} command to include url's (or filenames)
% \usepackage{url}
% 
% For multi page tables
\usepackage{longtable}
%
% For correct "List of Tables" entry:
% (1) find the file "longtable.sty"
% (2) find the line
%     "\addcontentsline{lot}{table}{\protect\numberline{\thetable}{#2}}}%"
% (3) change to
%     "\addcontentsline{lot}{table}{\protect\numberline{Table\nobreakspace \thetable}{#2}}}%"


% This package countains the \sout command (which you should never use!)
\usepackage[normalem]{ulem}

%(GLOSSARY) (see %(GLOSSARY) below)
%(EXPERIMENTAL) --- Remove / Comment out if you don't need a glossary
\usepackage{datatool}
\usepackage[nonumberlist,section=paragraph,automake]{glossaries}
\makeglossaries
\newacronym{sacd}{SACD}{snow cover area calculation and display}

\newacronym{psu}{psu}{practical salinity unit}

\newacronym{tp}{TP}{Tibetan Plateau}

\newacronym{qc}{QC}{quality control}

\newacronym{ims}{IMS}{Interactive Multisensor Snow and Ice Mapping System}

\newacronym{tsm}{TSM}{Tibetan Snow Man}

\newacronym{avhrr}{AVHRR}{Advanced Very High-Resolution Radiometer}

\newacronym{modis}{MODIS}{Moderate Resolution Imaging Spectrometer}

\newacronym{gms}{GMS}{Geostationary Meteorological Satellite}

\newacronym{metosat}{METEOSAT}{European Weather Satellite}

\newacronym{cmc}{CMC}{Common Mapping Client}

\newacronym{dat}{DAT}{NASA Sea Level Change Data Analysis Tool}

\newacronym{ssmi}{SSMI}{Special Sensor Microwave Imager}

\newacronym{dmsp}{DMSP}{Defense Meteorological Satellite Program}

\newacronym{amsu}{AMSU}{Advanced Microwave Sounding Unit}

\newacronym{nh}{NH}{Northern Hemisphere}

\newacronym{nsidc}{NSIDC}{National Snow and Ice Data Center}

\newacronym{laeap}{LAEAP}{Lambert azimuthal equal area projection}

\newacronym{foss}{FOSS}{free open source software}

\newacronym{t/s/p}{T/S/P}{temperature salinity pressure}

\newacronym{gdac}{GDAC}{Global Data Assembly Centres}

\newacronym{dac}{DAC}{Data Assembly Centres}

\newacronym{ftp}{FTP}{File Transfer Protocol}

\newacronym{loess}{LOESS}{local regression}

\newacronym{wmo}{WMO}{World Meteorological Organization}

\newacronym{aic}{AIC}{Akaike information criterion}

\newacronym{xml}{XML}{extensible markup language}

\newacronym{json}{JSON}{JavaScript object notation}

\newacronym{url}{URL}{Uniform Resource Locator}

\newacronym{api}{API}{application programming interface}


\newacronym{noaa}{NOAA}{National Oceanic and Atmospheric Administration}

\newacronym{nasa}{NASA}{National Aeronautics and Space Administration}

\newglossaryentry{filezilla}{
 name=FileZilla,
 description={Free software, cross-platform FTP application, consisting of FileZilla Client and FileZilla Server. Client binaries are available for Windows, Linux, and macOS.}}

\newglossaryentry{restfull}{
 name=RESTfull,
 description={REpresentational State Transfer, or RESTful, web services provide interoperability between computer systems on the Internet.}}

\newglossaryentry{aws}{
 name=AWS,
 description={Amazon Web Services (AWS) provides on-demand cloud computing platforms to individuals, companies and governments, on a paid subscription basis.}}
 
\newglossaryentry{hadoop}{
 name=Hadoop,
 description={Apache Hadoop is an open-source software framework used for distributed storage and processing of datasets of big data using the MapReduce programming model. It consists of computer clusters built from commodity hardware.}}

\newglossaryentry{rsync}{
 name=rsync,
 description={rsync is a utility for efficiently transferring and synchronizing files across computer systems, by checking the timestamp and size of files.}}

\newglossaryentry{leaflet}{
  name=leaflet,
  description={Open source JavaScript library used to build web mapping applications. Includes plugins for drawing objects, using different map projections, and using different map tiles.}}

\newglossaryentry{mvc}{
  name=MVC,
  description={Model–view–controller (MVC) is a software architectural pattern commonly used for developing user interfaces that divides an application into three interconnected parts.}}

\newglossaryentry{cron}{
  name=cron,
  description={A time-based job scheduler in Unix-like computer operating systems. Cron enables scripts to run at periodically at a fixed time.}}

\newglossaryentry{dataseries}{
 name=dataseries,
 description={A one-dimensional labeled array capable of holding any data type (integers, strings, floating point numbers, Python objects, etc.).}}

\newglossaryentry{dataframe}{
  name=DataFrame,
  description={2D Table like object with columns of potentially different types. Similar to a SQL table or MS Excel spreadsheet.}}


\newglossaryentry{dod}{
  name=document-oriented database,
  description={Is a type of NoSQL database, used to store semi-structured data, often in key-value schemas, such as XML or JSON. Unlike traditional relational databases, the data model in a document database is not structured in a table format of rows and columns. This allows flexibility in data structures and modeling.}}


\newglossaryentry{Iridium}{
  name=Iridium,
  description={Iridium Communications Inc. is a company that supplies Argo floats an antenna for data transmission.}}
\newglossaryentry{Ipython}{
  name=Ipython,
  description={IPython is an interactive command shell for the Python language.}}
  
\newglossaryentry{Jupyter}{
  name=Jupyter,
  description={A browser based notebook. Jupyter is a FOSS web application that allows, text, equations, and figure integration in a number of languages, includeing Python.}}
  
\newglossaryentry{express}{
  name=express.js,
  description={The standard server web application framework for Node.js. Used to make web-apps.}}

\newglossaryentry{node}{
  name=node.js,
  description={Node.js JavaScript run-time environment for executing JavaScript code server-side. Historically, javaScript was run on the client side only, with Node.js, javascript is run like any high-level interepreted language.}}
  
\newglossaryentry{mongodb}{
  name=MongoDB,
  description={MongoDB (from humongous) is document-oriented database program, classified as a NoSQL.  MongoDB uses JSON-like documents with schemas. MongoDB is developed by MongoDB Inc., and is published under a combination of the GNU Affero General Public License and the Apache License.}}

\newglossaryentry{WGS-84}{
  name=WGS-84,
  description={World Geodetic System established in 1984, is an Earth-centered and Earth-fixed terrestrial reference system and geodetic datum describing the Earth's shape, size, gravity and geomagnetic fields.}}

\newglossaryentry{.asc}{
  name=.asc,
  description={File that stores ascii characters}}

\newglossaryentry{HDF5}{
  name=HDF5,
  description={Hierarchical Data Format is a set of file formats designed to store and organize large amounts of data.}}

\newglossaryentry{NetCDF}{
  name=NetCDF,
  description={Network Common Data Form is a file storage system, denoted by .nc. It allows the creation, access, and sharing of array-oriented scientific data. The ARGO program releases data NetCDF on two FTP sites.}}
  
\newglossaryentry{ARGO}{
  name=ARGO,
  description={Array for Real-time Geostrophic Oceanography is a global array of 3,800 free-drifting profiling floats whos primary measurement includes temperature and salinity of the upper 2000 m of the ocean.}}

\newtheoremstyle{dtm}% name of the style to be used
  {0pt}% measure of space to leave above the theorem. E.g.: 3pt
  {0pt}% measure of space to leave below the theorem. E.g.: 3pt
  {\slshape}% name of font to use in the body of the theorem
  {0pt}% measure of space to indent
  {\bfseries}% name of head font
  {. }% punctuation between head and body
  {0pt}% space after theorem head
  {}% Manually specify head
\theoremstyle{dtm}

% The style of theorems and such that you want to use.  You can change
% the style by modifying the second argument (for example prepending a
% formatting command, e.g. \textsc{Theorem} which will make the
% headings come out as small caps rather then bold).
% 
% On second thought, don't change these formats as you are likely to
% incur the wrath of the Thesis Reviewer.
% 
\newtheorem{corollary}{Corollary}[chapter]
\newtheorem{definition}{Definition}[chapter]
\newtheorem{lemma}{Lemma}[chapter]
%\newtheorem{proof}{Proof}[chapter]
\newtheorem{proposition}{Proposition}[chapter]
\newtheorem{theorem}{Theorem}[chapter]


% Author name and the author name in upper case
% (FORMAT) Has to match university records, check if you have
% (FORMAT) full middle name, or middle initital on record.
\author{Yogesh Kohli}

% Title of the thesis (all in upper case), use \\ for line breaks as
% usual, you can use up to 4 lines and make sure to set the counter
% titlelines to the number of lines you used.
% 
% This is for the title page
% 
\title{Development of the EyesOnCrops \\
A new iOS app to visualize the NASA NDVI data}
% Number of lines in the title, without setting this the title page
% will not be formatted properly
\setcounter{titlelines}{3}

% Heading style title, the number of lines can be different here then
% in titlelines and in fact the thesis manual requires that this be at
% most 3 lines long so only put at most 2 pagebreaks here.  This is
% for the abstract pages and the signature page.
% 
% (FORMAT) Make sure that this title has the EXACT same words at the
% (FORMAT) title-page-title
% 
\titleheading{Development of the EyesOnCrops \\
A new iOS app to visualize the NASA NDVI data}


% Degree (with Concentration)
\degreeONE{Master of Science in Computer Science}
%\degreeTWO{Master of Science in Applied Mathematics}
%\degreeTHREE{Master of Science in Applied Mathematics}


% If you need to change the word 'Thesis' use \thesisname{Blah} and if
% you need to change the middle line between \degree and \degreein on
% the titlepage to something other then 'in' use \inofand{of} to use
% 'of' for instance.  (This should not be necessary)

% Dates
\gradyear{2018}
% (Format) Term Year 
\submitdate{Fall 2018}

% Your committee chair (don't include titles as per the manual)
\committeechair{Carl Eckberg}
\committeechairdept{Department of Computer Science}

% Co chair
\committeesecond{Samuel Shen}
\committeeseconddept{Department of Mathematics & Statistics}

% Third (usually different department) committee member
\committeethird{Roger Whitney}
\committeethirddept{Department of Computer Science}

% Third (usually different department) committee member
\committeefourth{Ming-Hsiang Tsou}
\committeefourthdept{Department of Geography}

\begin{document}

% Title page 
% (FORMAT) Mandatory for SDSU thesis
\maketitle

% Signature page
% (FORMAT) Mandatory for SDSU thesis
\makesignature

% Copyright page
% (FORMAT) Mandatory for SDSU thesis
\begin{copyrightpage}
  Copyright~\copyright~2018 \\
  by \\
  Yogesh Kohli
\end{copyrightpage}

% Dedication (make sure to format this correctly including a vspace
% (say \vspace{3in} or using vfill) to make it center on the page if
% desired, see the thesis manual) Or just delete this if you don't
% have a dedication
% 
% (FORMAT) Optional
\begin{dedication}
  \vspace{3in}
  \centering
  Dedicated to my parents, brother, sister-in-law, and my nephew, Prihaan for their consistent support all through this daring adventure.
\end{dedication}

% Epigraph (make sure to format this correctly, it will just be
% centered on the page, see the manual) Or just delete this if you
% don't have an epigraph
% 
% (FORMAT) Optional
\begin{epigraph}
Your time is limited, so don't waste it living someone else's life. Don't be trapped by dogma - which is living with the results of other people's thinking. Don't let the noise of others' opinions drown out your own inner voice. And most important, have the courage to follow your heart and intuition.
\begin{center}
    ― Steve Jobs
\end{center}
\end{epigraph}

% Here type the abstract of your thesis.
% (FORMAT) Mandatory for SDSU thesis
\begin{abstract}
  % This just inserts the the abstract.tex file
  % You insert your abstract in the space below.

This thesis develops a new iOS app named EyesOnCrops to visualize NASA's Normalized Difference Vegetation Index (NDVI) data in a user-friendly way. NDVI data, in the range between -1 and 1, represent the Earth’s surface vegetation by measuring the difference between near-infrared (for which vegetation strongly reflects) and red light (for which vegetation absorbs). A large positive NDVI value means good vegetation and hence is colored green. NDVI is a huge dataset of over 10 TB, with spatial resolution in ecodistrict and time resolution in 8 days. EyesOnCrops is the first technology that can conveniently and instantly deliver the NDVI data in both color maps and digital data. A user can use an iPhone, an iPOD, or an iPad to visualize the NDVI data at the spatial resolution levels of countries, states, or ecodistricts. Our tool will be extremely valuable to farmers, hedge fund traders and insurance business in addition to teachers and scientists. The thesis includes both the development theory and a user manual of EyesOnCrops. The first chapter reviews the available smartphones technology for spatial data visualization and gives brief introduction about app development methods. The second chapter describes NDVI applications and their significance and explains the EyesOnCrops user manual. EyesOnCrops has an important function of exporting data in the CSV format via email. The third chapter describes the toolkit - NDVI Dataset and its structure, Database Schema, Wireframes of the application and other software packages used for the creation of EyesOnCrops. The fourth chapter describes the process of using an app development toolkit to get NDVI data and converting the data into the format customizable for our mobile application. The fifth chapter made some statistical analysis of the NDVI data. In the end, this thesis also explains how the EyesOnCrops technology can be portable to the applications in other fields.



\end{abstract}

% Table of contents
% (FORMAT) Mandatory for SDSU thesis
\tableofcontents

% If you don't want a list of tables page, delete or comment out this
% line
% (FORMAT) ONLY delete this page if you have *no* tables
\listoftables
% If you don't want a list of figures page, delete or comment out this
% line
% (FORMAT) ONLY delete this page if you have *no* figures
\listoffigures
%(GLOSSARY) (see %(GLOSSARY) above)
%(EXPERIMENTAL) --- Remove / Comment out if you don't need a glossary
% Glossaries

\renewcommand*{\glsclearpage}{}
\begin{glossarypage}
  \centering
  \glsaddall\printglossary
\end{glossarypage}

% Your acknowledgments go here
% Or just delete this if you don't have acknowledgments
% (you should! - Suck up to your advisor and committee!!!)
\begin{acknowledgments}
Dr. Samuel Shen - for his unrelenting encouragement and support.
Alan Basist - for his expertise in NDVI data analysis.
Jeremy King - for his keen help in server side.
\end{acknowledgments}
%
% This is a diagnostic section to output the current font-selection is
% should be commented out... unless you're debugging font-selection,
% that is...
%
% Font parameters:
% \makeatletter
% \f@encoding -
% \f@family -
% \f@series -
% \f@shape -
% \f@size -
% \f@baselineskip -
% \tf@size -
% \sf@size -
% \ssf@size
% \makeatother
% 
% This includes body.tex
% 
\chapter{INTRODUCTION}
\label{chap:intro}

%intro here in one paragraph

The approach of enormous informational collections has put weight on the Mathematics and Statistics people group to reexamine how to apply conventional of investigation and inferencing. Both atmosphere and oceanographic science disciplines are feeling this information blast and are attempting to scale. Extraordinary measures of information are made accessible by administrative offices like \gls{noaa} and \gls{nasa}, and vigorously utilized by mainstream researchers to draw inductions. The customary perception and conveyance instruments can be enhanced to deal with a flood of new information conveyance. The methods used to envision and recover these information should be rethought. Worldwide expectations require current mobile application innovations for expert and novice to imagine and get information rapidly, precisely, and as effortlessly on their hands as could be expected under the circumstances.

Conventional strategies for parsing through a remote/nearby document of records, choosing significant documents, opening the documents and performing examination on a solitary \gls{pc} don't scale with enormous information. Researchers are investing more energy exploring through information as opposed to discovering disclosures. More awful still, beginners as youngsters, specialists, understudies and potential researchers get baffled and abandon utilizing these rich informational indexes. Toolboxes web applications and of course mobile applications are recommended in this proposal to alleviate this weight. Present day \gls{restfull} application configuration gives the network a mobile stick to enable them to swim through the huge information soil.


\section{Background and significance of visualizing spatial data using smart phones}

Government organizations deliver basic information about the country's populace, economy, administrations, agriculture and assets. These organizations are going under expanding weight, both societal and money related in nature, to create and actualize \gls{ict}, inside and encompassing their organizations, supporting another worldview of society and modernization, focused on electronic open administration. In these last years, a surprising accomplishment in the utilization of hand-held portable PCs — cell phones, tablet PCs, Personal Digital Assistants, and scratch pad — has been seen for information gathering in various fields.
Information accumulation is perceived as a standout amongst the most tedious, costly and mistake given errands in any information stock venture. Hand-held PCs hold the potential to lessen the calculated weight, cost, and blunder rate of paper-based strategies for information accumulation anyway there is an absence of proper, redid and specialized minimal effort arrangements. The eminent proceeding with development of the \gls{is} industry makes openings and difficulties for intriguing programming applications advancements and usage. In this sense, Geographic Information Systems \gls{gis} and the electronic open organization administrations come into a typical circle. \gls{gis} information frameworks effectively store and control, join, and interrelate spatial genuine articles (e.g. political limits, streets, offices areas). Spatial information verbalized with other information sources gives proficient intends to arranging, basic leadership, and administration numerous parts of financial exercises which take advantage of a spatial measurement. Portability is a progressive wonder and establishes the most imperative current pattern in \gls{it}, uniquely, the classification of ease arrangements.

Portable figuring frameworks and equipment are changing the manner in which versatile mapping innovation is being utilized by moving \gls{gis} from the work area into the client's hands, giving adaptability in information securing, information precision and uprightness — approval progressively lessening blunders and process costs — more data with significantly less time and exertion, quicker correspondence conventions, and high profitability, making the portability a tempting part of \gls{gis}.

%
\begin{figure}[ht]
  \centering
  \begin{minipage}{4.5in}
    \includegraphics[width=\linewidth]{ims_data.jpg}
    \caption{ \label{fig:nat_ice} Sea and Lake Ice coverage of IMS data using 4 KM resolution. Ice coverages are calculated using a three day running mean from March until September each year. Blue border displays maximum and minimum values for the season. Areas are calculated using the Lambert Azimuthal Equal Area Projection with a WGS84 Datum. \cite{nat_ice}}
  \end{minipage}
\end{figure}
%


\subsection{Past work}
Talking about mobile technologies, Applications began as a stripped down, single-work program to keep running on the telephone and some have kept on being only that. Be that as it may, with advances in equipment and programming the pattern has moved back to having applications accomplish more. This has enabled individuals to supplant a few bits of tech with a solitary cell phone or tablet and, simultaneously, made the information from that gadget a lot more entire. What's more, that information, as opposed to the applications themselves, is the place the genuine world-changing force lies.

Portable application configuration is a solid help for understudy focused registering. By including visual and spatial information in a portable application, understudies can build up a 3-D execution which can furnish the versatile application clients with a virtual ordeal. The improvement of a versatile application for a chronicled cemetery gives a case of how to coordinate database data with visual and spatial information to accomplish s virtual experience. The contextual analysis introduced here, utilizing both Android and \gls{iOS} gadgets, incorporates three sections. At first, a current database was changed over for versatile application get to. This was trailed by plan coordination in help of the coveted versatile application highlights. At last, the incorporation of a picture display, with visual and spatial components, coordinated with the portable application, brought about a convincing versatile application, giving a virtual copy of a genuine visit to the recorded site.

%
\begin{figure}[ht]
  \centering
  \begin{minipage}{4.5in}
    \includegraphics[width=\linewidth]{ims_data.jpg}
    \caption{ \label{fig:nat_ice} Sea and Lake Ice coverage of IMS data using 4 KM resolution. Ice coverages are calculated using a three day running mean from March until September each year. Blue border displays maximum and minimum values for the season. Areas are calculated using the Lambert Azimuthal Equal Area Projection with a WGS84 Datum. \cite{nat_ice}}
  \end{minipage}
\end{figure}
%


\subsection{Current scenario}

Notwithstanding late advances in pen-as well as contact empowered portable gadgets and the quick appropriation of these gadgets in ordinary life, we are a long way from utilizing the maximum capacity of cell phones in fulfilling the developing interest for visual access to information. Despite the fact that the plan space for versatile  information perception is developing out of ordinary practice [14], concentrated research endeavors have not yet risen.





\subsection{Future work for quick visualization and analysis of big data}



\section{Motivation of the VACYD research: crop yield data visualization}



\section{A short summary of the app development method and results}



\begin{figure}[ht]
  \centering
  \begin{minipage}{4.5in}
    \includegraphics[width=\linewidth]{commTS.png}
    \caption{ \label{fig:commTS} Satalite Communication Type by month. Iridium floats also use the GPS system.}
  \end{minipage}
\end{figure}

\begin{figure}[ht]
  \centering
  \begin{minipage}{4.5in}
    \includegraphics[width=\linewidth]{operation_park_profile.jpg}
    \caption{ \label{fig:argo_cycle} Detail of one profile cycle\cite{argo}}
  \end{minipage}
\end{figure}

\section{Literature review}
\chapter{NDVI-MAP FUNCTIONS AND USER GUIDE}
\label{chap:ndvi & it's user guide}

\section{What can app do?}

Application enables the user to envision \gls{ndvi} mean information stored in the database. It likewise gives the usefulness of sending out the information into \gls{csv} record and messaging it to anyone that needs it. This causes the user to visualize information utilizing diverse levels such as Country, State and Eco-District wise.

\section{A user guide/manual}

A guide is a short reference to some particular aspects of a software product. The user guide of the application is explained below.

\begin{itemize}
    \item \textbf{Downloading the app} \\
    At the presen time, the app is in process of being submitted to App Store for review. Once it is approved by Apple, anyone can utilize the "EyesOnCrops" application by downloading it from the Apple Store. 
    
    \begin{itemize}
        \item Open the App Store and type "EyesOnCrops" in search bar.
        \item Download and introduce the "EyesOnCrops" application on the device.
        \item Once installed, it will look like Figure~\ref{fig:app_icon_screen}.
        
        \begin{figure}[H]
            \centering
            \includegraphics[width=0.50\linewidth]{figures/ch2/app_icon_screen.png}
            \caption{\label{fig:app_icon_screen} iPhone screen after downloading the app}
        \end{figure}
    \end{itemize}
    
    \item \textbf{Getting in the app} \\
    By tapping on app icon on Figure~\ref{fig:app_icon_screen}, it will lake the user to landing screen of the app, also known as the main screen. This screen gives two options, which are shown in Figure 2.2.
     
        \begin{figure}[H]
            \centering
            \includegraphics[width=0.50\linewidth]{figures/ch2/main_screen.png}
            \caption{\label{fig:main_screen} Main / Landing screen of the app}
        \end{figure}

    \textbf{1. Register} \\
    \textbf{2. Login} \\
    
    By selecting "\textbf{Register}" in Figure~\ref{fig:main_screen}, the user is taken to registration process. This is divided into 5 steps which are explained below.
    
        \begin{figure}[H]
            \centering
            \includegraphics[width=0.50\linewidth]{figures/ch2/register_personal.png}
            \caption{\label{fig:register_personal} Personal details screen - Register process}
        \end{figure}
  
        
        \begin{figure}[H]
            \centering
            \includegraphics[width=0.50\linewidth]{figures/ch2/register_contact.png}
            \caption{\label{fig:register_contact} Contact details screen - Register process}
        \end{figure}
     
        \begin{figure}[H]
            \centering
            \includegraphics[width=0.50\linewidth]{figures/ch2/credentials_setup.png}
            \caption{\label{fig:credentials_setup} Setup credentials screen - Register process}
        \end{figure}
       
        
        \begin{figure}[H]
            \centering
            \includegraphics[width=0.50\linewidth]{figures/ch2/register_location.png}
            \caption{\label{fig:register_location} Location details screen - Register process}
        \end{figure}
 
        \begin{figure}[H]
            \centering
            \includegraphics[width=0.50\linewidth]{figures/ch2/purpose_app.png}
            \caption{\label{fig:purpose_app} Purpose of using the app screen - Register process}
        \end{figure}

    
    By selecting "\textbf{Login}" in Figure~\ref{fig:main_screen}, the user is taken to the screen that gives options to enter the application which is shown in Figure~\ref{fig:loginOptions}.
    
    \begin{figure}[H]
            \centering
            \includegraphics[width=0.50\linewidth]{figures/ch2/loginOptions.png}
            \caption{\label{fig:loginOptions} Login options in the app}
    \end{figure}
    
     \textbf{1. Login via Social Accounts}
     \begin{itemize}
         \item Facebook login
         \item Gmail login
     \end{itemize}
   
     \textbf{2. Login via Email} \\
    This screen requires the user credentials to enter the application. The screen has appeared in Figure~\ref{fig:login_email}.
     
     \begin{figure}[H]
            \centering
            \includegraphics[width=0.50\linewidth]{figures/ch2/login_email.png}
            \caption{\label{fig:login_email} Login via email}
    \end{figure}
    
    
    \item \textbf{Home} \\
    By login in the application, the user arrives to the home screen which fundamentally has everything to use the application.
    
    \begin{figure}[H]
            \centering
            \includegraphics[width=0.5\linewidth]{figures/ch2/home.png}
            \caption{\label{fig:home_screen} Home page - landing page after login}
    \end{figure}
    

    \item \textbf{Slide out menu} \\
    Slide out menu has been utilized to give the application an extremely pleasant look in light of keeping the ease of use while exploring in the application. It has been shown in Figure~\ref{fig:side_menu}.
    
     \begin{figure}[H]
            \centering
            \includegraphics[width=0.50\linewidth]{figures/ch2/side_menu.png}
            \caption{\label{fig:side_menu} Slide out menu after selecting menu option on Home screen}
    \end{figure}
    
    
   \item \textbf{Filter screens} \\
    Tapping on filter button on home screen takes the user to Figure~\ref{fig:filter_screen}.
    
    \begin{figure}[H]
            \centering
            \includegraphics[width=0.50\linewidth]{figures/ch2/filter_screen.png}
            \caption{\label{fig:filter_screen} Filter screen after selecting filter option on Home screen}
    \end{figure}
    
    Now, there are various ways to filter the data which are mentioned below.
   
    \begin{itemize}
        \item \textbf{Year list}
        
        Year and Date wise selections on filter screen takes the user to the screen shown in Figure~\ref{fig:years_list_screen}.
        
         \begin{figure}[H]
            \centering
            \includegraphics[width=0.50\linewidth]{figures/ch2/year_list.png}
            \caption{\label{fig:years_list_screen} Filter - Year list screen}
         \end{figure}
   
        \item \textbf{Admin level list}
        
        Admin Level tap takes the user to the screen shown in Figure~\ref{fig:level_list_screen}.
        
         \begin{figure}[H]
            \centering
            \includegraphics[width=0.50\linewidth]{figures/ch2/level_list.png}
            \caption{\label{fig:level_list_screen} Filter - Admin level list screen}
        \end{figure}

     \item \textbf{Color scheme list}
     
     Color scheme tap takes the user to the screen shown in Figure~\ref{fig:color_scheme}.
        
        \begin{figure}[H]
            \centering
            \includegraphics[width=0.50\linewidth]{figures/ch2/color_scheme.png}
            \caption{\label{fig:color_scheme} Filter - Color palette  options screen}
        \end{figure}
        
    \end{itemize}

\end{itemize}




\chapter{DATASET, DATABASE AND DESIGN OF THE NDVI APPLICATION}
\label{chap:dataset & database}

\section{NDVI dataset and its significance}

\subsection{About MODIS}

\newcommand{\MYhref}[3][blue]{\href{#2}{\color{#1}{#3}}}%

\centerline{\MYhref{https://modis.gsfc.nasa.gov/}{NASA's MODIS website}}

\gls{modis} is a key instrument on board the Terra (initially known as EOS AM-1) and Aqua (initially known as EOS PM-1) satellites. Land's circle around the Earth is coordinated with the goal that it goes from north to south over the equator early in the day, while Aqua disregards south to north the equator toward the evening. Land \gls{modis} and Aqua \gls{modis} are seeing the whole Earth's surface each 1 to 2 days, gaining information in 36 ghastly groups, or gatherings of wavelengths (see \gls{modis} Technical Specifications). These information will enhance our comprehension of worldwide elements and procedures happening on the land, in the seas, and in the lower climate. \gls{modis} is assuming a fundamental job in the improvement of approved, worldwide, intuitive Earth framework models ready to foresee worldwide change precisely enough to help arrangement creators in settling on quality choices concerning the security of our condition. \\

\centerline{\textbf{Terra vs. Aqua}}

According to \gls{nsidc} distributes \gls{modis} data from both the Terra and Aqua satellites.

Figure 3.1 shows the difference between the two satellites. For the application, we have worked with Aqua satellite's data.

 \begin{figure}[H]
            \centering
            \includegraphics[width=1.0\linewidth]{figures/ch3/satellites.png}
            \caption{\label{fig:modis_satellites_difference} Difference between MODIS satellites}
    \end{figure}

\subsection{Overview of NDVI Dataset}

\gls{modis} vegetation indices, delivered on 16-day interims and at various spatial goals, give steady spatial and worldly correlations of vegetation covering greenness, a composite property of leaf territory, chlorophyll and shade structure. Two vegetation lists are gotten from environmentally redressed reflectance in the red, close infrared, and blue wavebands; the standardized contrast vegetation list \gls{ndvi}, which gives coherence \gls{noaa}'s AVHRR \gls{ndvi} time arrangement record for verifiable and atmosphere applications, and the upgraded vegetation list (EVI), which limits covering soil varieties and enhances affectability over thick vegetation conditions. The two items all the more successfully describe the worldwide scope of vegetation states and procedures. 

As of now, NASA has the website to utilize the GIMMS Global Agricultural Monitoring (GLAM) System to see MODIS NDVI symbolism and recover MODIS NDVI time arrangement information. The framework gives close ongoing and science quality Terra and Aqua MODIS 8-day composited, worldwide NDVI datasets. These datasets are gotten from the Collection 6 MOD09 and MYD09 surface reflectance items which are given by NASA/GSFC/EOSDIS LANCE and NASA/GSFC MODAPS.

The GIMMS MODIS GLAM System is produced and given by the NASA/GSFC/GIMMS bunch for the USDA/FAS/IPAD Global Agricultural Monitoring venture. The USDA/FAS/IPAD mission is to give objective, auspicious, and consistent appraisal of the worldwide agrarian generation standpoint and conditions influencing worldwide sustenance security. You can see the website at the url mentioned below. \\
\centerline{\MYhref{https://glam1.gsfc.nasa.gov//}{NASA's NDVI website}}

Please see the figures 3.2 and 3.3 for user guide of the website.

    \begin{figure}[H]
            \centering
            \includegraphics[width=1.0\linewidth]{figures/ch3/nasa_website_1.png}
            \caption{\label{fig:nasa_website_1} NASA's NDVI website user guide part-I}
    \end{figure}
    
    
    \begin{figure}[H]
            \centering
            \includegraphics[width=1.0\linewidth]{figures/ch3/nasa_website_2.png}
            \caption{\label{fig:nasa_website_2} NASA's NDVI website user guide part-II}
    \end{figure}    


The vegetation records are recovered from day by day, air redressed, bidirectional surface reflectance. The VI's utilization a \gls{modis}-particular compositing strategy dependent on item quality confirmation measurements to evacuate low quality pixels. From the staying great quality VI esteems, an obliged see edge approach at that point chooses a pixel to speak to the compositing time frame (from the two most elevated \gls{ndvi} esteems it chooses the pixel that is nearest to-nadir). Since the \gls{modis} sensors on board Terra and Aqua satellites are indistinguishable, the VI calculation creates every 16-day composite eight days separated (staged items) to allow a higher worldly goals item by joining the two information records. The \gls{modis} VI item suite is currently utilized effectively in all environment, atmosphere, and regular assets administration contemplates and operational research as exhibited by the consistently expanding group of associate distributions. \\

\centerline{\textbf{How do you calculate NDVI?}}

\textbf{\[ NDVI = \frac{NIR - RED}{NIR + RED} \ \ \ 
\ \ \]}

\centerline{NIR: \gls{nir}}
\centerline{RED: \gls{red}}

Solid vegetation (chlorophyll) reflects more close \gls{nir} and green light contrasted with different wavelengths. In any case, it retains more red and blue light. 

This is the reason our eyes consider vegetation to be the shading green.

Let's look at the example shown in the fig 3.4.

    \begin{figure}[H]
            \centering
            \includegraphics[width=0.5\linewidth]{figures/ch3/ndvi-example.png}
            \caption{\label{fig:ndvi_example} NDVI Example - Courtsey: NASA}
    \end{figure}

Data is normally stored in a \gls{gdac} \gls{ftp} server, at a US-based server at \url{https://gimms.gsfc.nasa.gov/MODIS/}.

The information is thought to be adequately quality controlled and in a steady arrangement. Stage and profile name is the main distinguishing proof. This information structure does not permit regular determination or spatial choice: a key element to any atmosphere informational index. The FTP server is intended to go about as a chronicle. Researchers are urged to download the information locally for their utilization's, a procedure that depends on noteworthy space learning of the FTP website. Barely any individuals will experience the inconvenience of understanding the FTP structure, less still will set aside opportunity to make the procedure less demanding for other people.

\subsection{Significance}

The term \gls{ndvi} in the farming business has positively increased more mindfulness of late because of the developing ubiquity of little unmanned elevated vehicles. 

\gls{ndvi} is positively not another child on the square and with regards to social affair and preparing this data, rural experts have known and been utilizing this information for quite a long time. Anyway already assembling this information may have been tedious, awkward, not exceptionally precise and costly. 

As innovation enhances and the methods in which we would now be able to catch this information, agriculturists are beginning to take this basic and successful information all the more genuinely by hoping to join this technique into their product administration methodology. 

Information is enter in exactness farming, knowing your product's well being is a certain something yet really pre-empting the state of harvest's well being, endorsing compost application, distinguish any potential sickness and precisely assessing yield puts the control in your grasp consequently making you all around educated to settle on the correct business choices as and when required.

\gls{ndvi} is valuable for evaluating the well being and thickness of vegetation. \gls{ndvi} esteems close to 0 show extremely inadequate vegetation. Thick vegetation is demonstrated by \gls{ndvi} esteems moving toward 1. By utilizing a period arrangement of \gls{ndvi} perceptions, one can look at the elements of the developing season and screen wonders, for example, dry spell. A full supplement of information has been finished from the earliest starting point of Terra satellite activity to the present and is accessible for download in peruse and 250m goals.


\section{Wireframes Designs}

A versatile application wireframe, otherwise called a page schematic or screen outline, is a visual guide that speaks to the skeletal structure of an application. Wireframes are made to arrange components to best achieve a specific reason.

In other words, they are the skeleton of any product. This skeleton is a two-dimensional delineation of a page's interface that demonstrates the dividing of components on the page, how content is organized, what functionalities are accessible, and how clients will cooperate with the site. They likewise assume an essential job in interfacing data engineering to the visual parts of the outline by indicating pathways between the different pages. Wireframes are purposefully drained of shading, illustrations and adapted textual styles. The tool used for creating wireframes here is \textbf{Axure RP 8}. \\

Reasons for using wireframes are mentioned below:

\begin{itemize}
    \item Wireframes enable you to outline the usefulness of the pages, get issues early, and spare time on corrections later. It is considerably less excruciating to roll out improvements to a wireframe than to a high devotion mockup with heaps of plan components. \\
    
    \item Wireframes are an incredible method to organize content by uncovering space requirements and planning the pecking order of components on the page. Having the open door right off the bat to envision the chain of command of your pages and start outwardly showing the space requirements will spare you a ton of time later when you start adapting the pages and filling them with substance. \\
\end{itemize}

\centerline{\textbf{Some of the Wireframes designs are shown below in figure 3.5, 3.6 and 3.7}} 

    \begin{figure}[H]
            \centering
            \includegraphics[width=0.5\linewidth]{figures/ch3/wireframe_1.png}
            \caption{\label{fig:wireframe_1} Wireframe - Registration process}
    \end{figure}
  
  
  
    \begin{figure}[H]
            \centering
            \includegraphics[width=0.5\linewidth]{figures/ch3/wireframe_2.png}
            \caption{\label{fig:wireframe_2} Wireframe - Login and forgot password process}
    \end{figure}
    
    \begin{figure}[H]
            \centering
            \includegraphics[width=0.5\linewidth]{figures/ch3/wireframe_3.png}
            \caption{\label{fig:wireframe_3} Wireframe - Main screen and home screen}
    \end{figure}


\section{Database Architecture}

\textbf{Definition}

An organized arrangement of information held in a \gls{pc}, particularly one that is available in different ways.


Database structure was outlined subsequent to planning \gls{uml} Diagrams in order to improve comprehension of the structure.

In the \gls{uml}, an utilization case chart can condense the subtle elements of your framework's clients (otherwise called on-screen characters) and their communications with the framework. Figure 3.8 shows the use case diagram of the app.

    \begin{figure}[H]
            \centering
            \includegraphics[width=1.0\linewidth]{figures/ch3/usecase.png}
            \caption{\label{fig:wireframe_3} Use case diagram of the app}
    \end{figure}

As you can see in the above figure, it clearly shows the actions which user can interact with in the app.

On the other hand, Class diagrams are a standout amongst the most valuable kinds of outlines in \gls{uml} as they plainly delineate the structure of a specific framework by demonstrating its classes, characteristics, activities, and connections between articles. With our  \gls{uml} graphing programming, making these charts isn't as overpowering as it may show up. This guide will demonstrate to you generally accepted methods to comprehend, plan, and make your own class outlines. Figure 3.9 will give you better understanding how class diagrams can be used to make the product better.

    \begin{figure}[H]
            \centering
            \includegraphics[width=0.8\linewidth]{figures/ch3/classdiagram.png}
            \caption{\label{fig:wireframe_3} Class diagram of the app}
    \end{figure}
    
    \newpage

 \centerline{\textbf{Database Structure}}    
  PostgreSQL database has been used for the project, which is a general purpose and object-relational database management system. 
  Database tables and their entities are shown in the figure 3.10.
  
  \begin{figure}[H]
            \centering
            \includegraphics[width=1.0\linewidth]{figures/ch3/database_structure.png}
            \caption{\label{fig:database_structure} Database tables}
    \end{figure}
    
    





s\chapter{DEVELOPMENT OF THE iOS APP}
\label{chap:development of the app}

The development of the iOS app is divided into two areas which are :

\begin{itemize}
    \item Front-end
        \begin{itemize}
            \item App designing
            \item Mobile App screens and their significance
        \end{itemize}

    \item Back-end
        \begin{itemize}
            \item Process of getting data from NASA's server
            \item Web services required for \gls{json} parsing between database and the front-end
        \end{itemize}
    
     \item API Usage
        \begin{itemize}
            \item Major APIs used
        \end{itemize}    
\end{itemize}

\section{FRONT-END OF THE APP}

\subsection{App Designing}
    
\centerline{\textbf{Xcode - IDE}}

\centerline{Xcode 9.3 has been used for the development of the app}

\textbf{Xcode} is an Integrated Development Environment, which implies it pulls every one of the instruments expected to create an application (especially a content tool, a compiler, and a manufacture framework) into one programming bundle instead of abandoning them as an arrangement of individual devices associated by contents. Xcode is Apple's authentic \gls{ide} for \gls{mac} and \gls{iOS} engineers. It was initially known as Project Builder in the NeXT days, and renamed to Xcode some place around Mac OS X 10.3 or 10.4. By adaptation 4, Apple had collapsed in the sidekick Interface Builder program so there was just a single application package; the plan of the program hasn't changed a ton from that point forward, albeit clearly the instruments are refreshed routinely. \\

%add to this biblipography as my copying from apple's website.

Apple provides Built-In Interface Builder in the \gls{ide} for designing.
According to Apple's website, The Interface Builder editor within Xcode makes it simple to design a full user interface without writing any code. Simply drag and drop windows, buttons, text fields, and other objects onto the design canvas to create a functioning user interface. \\

Because Cocoa and Cocoa Touch are built using the Model-View-Controller pattern, it is easy to independently design your interfaces, separate from their implementations. User interfaces are actually archived Cocoa or Cocoa Touch objects (saved as .nib files), and \gls{macOS} and \gls{iOS} will dynamically create the connection between \gls{ui} and code when the app is run. \\

%add to this biblipography as my copying from apple's website.
%https://developer.apple.com/library/archive/documentation/General/Conceptual/Devpedia-CocoaApp/Storyboard.html
As you know, i have used built in feature of Xcode for designing. Apple also provides one more tool for visualization of the design which is \textbf{Storyboard}. \\ \\
According to Apple, A storyboard is a visual representation of the user interface of an \gls{iOS} application, showing screens of content and the connections between those screens. A storyboard is composed of a sequence of scenes, each of which represents a view controller and its views; scenes are connected by segue objects, which represent a transition between two view controllers. \\

Xcode provides a visual editor for storyboards, where you can lay out and design the user interface of your application by adding views such as buttons, table views, and text views onto scenes. In addition, a storyboard enables you to connect a view to its controller object, and to manage the transfer of data between view controllers. Using storyboards is the recommended way to design the user interface of your application because they enable you to visualize the appearance and flow of your user interface on one canvas. Figure 4.1 shows the storyboard of the app. 

Some of the advantages of using Storyboard are mentioned below:-
\begin{itemize}
    \item It's a container for all your Scenes (View Controllers, TabBar Controllers and more).
    \item A director of associations and transitions between these scenes (called Segues).
    \item It gives you the chance to see what My view will look like at runtime without running your application.
\end{itemize}

\newpage

    \begin{figure}[H]
            \centering
            \includegraphics[width=\linewidth]{figures/ch4/storyboard_2.png}
            \caption{\label{fig:wireframe_3} Storyboard of the app}
    \end{figure}
    
\subsection{Mobile App screens and their significance}

\subsubsection{Sign-up via Email}

Sign up process is a one time process which enables you to enter the application out of the blue. It likewise gives us the chance to include the client in our database for future logins and additionally the record.

Sign up via email process consists of 5 screens which are listed below:-

\newpage

\begin{itemize}

    \item Personal details
    
    \begin{figure}[H]
            \centering
            \includegraphics[width=0.4\linewidth]{figures/ch2/register_personal.png}
            \caption{\label{fig:wireframe_3} Register Part-I Personal detail screen}
    \end{figure}
    
    Here in the figure 4.2, mandatory fields are First name, Last name and year of birth where as we have kept middle name as an optional because not everyone has one. \\
    
    UIPickerView is used for selection of year of birth, According to Apple's documentation, UIPickerView is a view that uses a spinning-wheel or slot-machine metaphor to show one or more sets of values.
    
    Figure 4.3 shows the image of the pickerview designed in the app.
    
    \begin{figure}[H]
            \centering
            \includegraphics[width=0.4\linewidth]{figures/ch4/pickerview.png}
            \caption{\label{fig:wireframe_3} UIPickerView for year selection}
    \end{figure}
    
    \newpage
    
    \item Contact details
    
    \begin{figure}[H]
            \centering
            \includegraphics[width=0.5\linewidth]{figures/ch2/register_contact.png}
            \caption{\label{fig:wireframe_3} Register Part-II Contact detail screen}
    \end{figure}
    
    Figure 4.3 represents contact detail screen in which both the fields i.e Email and Phone are mandatory for record purpose and to avoid fake entries. User with unique email and phone can register only once. It's a good point to note that both the fields have been verified syntactically here.
    
    \newpage
  
    \item Location details
    
    \begin{figure}[H]
            \centering
            \includegraphics[width=0.5\linewidth]{figures/ch2/register_location.png}
            \caption{\label{fig:wireframe_3} Register Part-III Location detail screen}
    \end{figure}
    
    This screen makes use of Apple's Core Location framework which allows developers to obtain current geographic location of device.
    
    In this screen, user has two possible choices for entering his location, he can enter the area physically or can simply tap on pickup current area which will eventually get the present location and display it in text area showed. Despite everything he can still alter the area if he wants to.
    
    \newpage
    
    \item Setup Credentials
    
    \begin{figure}[H]
            \centering
            \includegraphics[width=0.5\linewidth]{figures/ch2/credentials_setup.png}
            \caption{\label{fig:wireframe_3} Register Part-IV Credentials setup screen}
    \end{figure}
    
    As you can see in figure 4.6, it allows you to setup password for your secure log in. Point to note here is that minimum length of password has been set 6 digits.
    
    
    \newpage
    
    \item Purpose of using the app
    
     \begin{figure}[H]
            \centering
            \includegraphics[width=0.5\linewidth]{figures/ch2/purpose_app.png}
            \caption{\label{fig:wireframe_3} Register Part-V Purpose of using the app screen}
    \end{figure}
    
    This screen is really important for us to know what and why people wants to use this app. Moreover, it has additional check mark for privacy policy and terms of use to which every user must agree before going further. It also protects us from some bot attack.
    
    \newpage
 
\end{itemize}


\subsubsection{Login process}

User can log in in the app via certain ways:-

\begin{itemize}
    \item Social Login
    
    \begin{itemize}
        \item Facebook Login
        \item Google Signin
    \end{itemize}
    
    \item Login via email
    
\end{itemize}

\subsubsection{Home screen}
\subsubsection{Sliding Menu}
\subsubsection{Filter process and its sub-screens}

\section{API usage}
\section{Softwares used in making of the app}

Visualization Apps like Argovis have served two purposes. The first purpose is to serve the community; the second is to expand the community. Global data sets in the oceanographic and earth science disciplines could potentially benefit from such apps, with minimal cost to their endeavors. The architecture of Argovis is designed for high traffic demands at low computation by placing visualization loads on the client side. The server side acts as a database and web server only; this allows a relatively high traffic web app be hosted and maintained at a low cost. Moreover, the software used is open source. 

Governmental agencies NOAA and NASA are required to release their data. The amount of data gathered and released is now on the petabyte scale. Without accessibility, this amounts to little more than an archive, accessible only to domain experts. This thesis proposes that agencies and groups design their data interfaces with a user interface in mind. Argovis is designed around the following data acquisition procedure: 
\begin{enumerate}
  \item browse datasets. 
  \item assess its relevance to their needs.
  \item visualize/select their region of interest.
  \item download data in a format they can use.
\end{enumerate}
An additional consideration would be to include a discussion forum to the web app, similar to \url{www.kaggle.com} \cite{kaggle} competitions, where each dataset has its discussion forum, where users post their code, figures and results in \gls{Jupyter} notebooks. Forums allow outside users to contribute to the project. As the institutions themselves may be limited in manpower and funding, they may rely on forums for additional code, tutorials or discussion. For example, Argovis has an API in Python, but users proficient in other scientific languages can write APIs in other languages, such as R, Matlab, and Julia. A forum section of the site works as a source of feedback for the development team. Questions, requests, complaints can be posted and be resolved/answered by either other users or the developers themselves.

The scientific community can become bogged down in data storage problem. Open, community-driven web applications to big data visualization and analysis can help users float on this sea of data.
\chapter{APP AS A DATA ANALYSIS TOOL}
\label{chap:analysis_tool}

\textbf{Data analysis} is a procedure used to examine, clean, change and rebuild information with a view to reach to a specific decision for a given circumstance. Information investigation is normally of two sorts: subjective or quantitative. The sort of information directs the technique for examination. In subjective research, any non-numerical information like content or individual words are broke down. Quantitative examination, then again, centers around estimation of the information and can utilize insights to help uncover results and ends. The outcomes are numerical. At times, the two types of examination are utilized as an inseparable unit. For instance, quantitative investigation can help demonstrate subjective ends. \\

Spatial information investigation is concerned about that part of information examination where the land referencing of articles contains imperative data. This chapter talks about the ways this app can be used as an analysis tool directly or indirectly. \\

\section{Data downloading}

\textbf{Downloading} is defined as the transfer of data from server to your system or belonging. In other words, it is defined as the transmission of data from one machine to another. The significant advantage of downloading is that it gives you the full power over the information with the goal that you can utilize that information. \\

The app gives you the feature of exporting any admim level (0, 1 or 2) data in \gls{csv} format via email option so that it can be used in any form of spatial analysis.

Process of data downloading is described below.

\begin{itemize}
    \item Select year, date and tap on country to see the data and then tap on any region. Once you do that, you will see two buttons on both top corners of the screen. Figure 5.1 shows the two buttons which appears after having valid data on globe / map. \\
    
      \begin{figure}[H]
            \centering
            \includegraphics[width=0.25\linewidth]{figures/ch5/buttons.png}
            \caption{\label{fig:buttons} Home screen with two buttons on top corners}
        \end{figure}
     
    \item Left button in figure 5.1 corresponds to Info button which shows information about the current data that is being displayed. Figure 5.2 shows the view appears on tapping of information button. \\
    
      \begin{figure}[H]
            \centering
            \includegraphics[width=0.25\linewidth]{figures/ch5/info_view.png}
            \caption{\label{fig:info_button} Information button action}
        \end{figure}
       
     \item Right button in figure 5.1 corresponds to export button which prompts you for exporting current data which is visible on globe / map. Figure 5.3 shows the export view on tapping of export button. \\
     
     \begin{figure}[H]
            \centering
            \includegraphics[width=0.25\linewidth]{figures/ch5/export_view.png}
            \caption{\label{fig:info_button} Export button action on home screen}
    \end{figure}
    
    On selecting yes in the figure 5.3, it then converts the raw \gls{json} data to a valid \gls{csv} format and attaches it as a file on the MFMailComposeViewController. \\
    
    %bibliography here add apple url link of definition
    
    According to Apple, It's a standard interface for managing, editing, and sending an email message in \gls{iOS} app. Figure 5.4 shows the MFMailComposerViewController view which helps user to email the \gls{csv} file.
    
    \begin{figure}[H]
            \centering
            \includegraphics[width=0.25\linewidth]{figures/ch5/export_view.png}
            \caption{\label{fig:info_button} Email controller after selecting yes on export action screen}
    \end{figure}
    
\end{itemize}

\newpage

\section{Computing mean, Standard Deviation, histograms}

\begin{itemize}
    \item \textbf{Mean} \\
    %biblioography https://store.fmi.uni-sofia.bg/fmi/statist/education/Virtual_Labs/freq/freq2.html
    According to Wikipedia, The mean of a data set is simply the arithmetic average of the values in the set, obtained by summing the values and dividing by the number of values. Recall that when we summarize a data set in a frequency distribution, we are approximating the data set by "rounding" each value in a given class to the class mark. With this in mind, it is natural to define the mean of a frequency distribution by figure 5.5.
    
     \begin{figure}[H]
            \centering
            \includegraphics[width=0.25\linewidth]{figures/ch5/mean_formula.png}
            \caption{\label{fig:info_button} Mean formula}
    \end{figure}
    
    \item \textbf{Standard Deviation} \\
   It is defined as the amount computed to show the degree of deviation for a gathering all in all.
    
     \begin{figure}[H]
            \centering
            \includegraphics[width=0.25\linewidth]{figures/ch5/standard_deviation.png}
            \caption{\label{fig:info_button} Standard deviation formula}
    \end{figure}
    
\end{itemize}

\centerline{\textbf{Why use mean and standard deviation for analysis?}}

The main purpose of using mean and standard deviation as a parameter for analysis is that the Normal Curve discloses to us that numerical data will be disseminated in an pattern around a normal line whereas Standard deviation is viewed as the most valuable list of variability. It is a solitary number that discloses to us the inconstancy, or spread, of an appropriation (gathering of scores).

We have investigated \gls{ndvi} and Anomaly a portion of the nations just to demonstrate that with the sent out information through the app, client can do these things:

\newpage

\begin{itemize}
    \item \textbf{Mean NDVI \& Anomaly distribution over the years}
    
    \begin{figure}[!htb]
        \begin{minipage}{0.5\textwidth}
            \centering
            \includegraphics[width=1.0\linewidth]{figures/ch5/Mean/AUSTRALIA_mean.png}
            \caption{Mean - Australia}\label{Fig:AUSTRALIA_mean}
        \end{minipage}\hfill
        \begin{minipage}{0.5\textwidth}
            \centering
            \includegraphics[width=1.0\linewidth]{figures/ch5/Mean/BRAZIL_mean.png}
            \caption{Mean - Brazil}\label{Fig:BRAZIL_mean}
        \end{minipage}
    \end{figure}
    
     \begin{figure}[!htb]
        \begin{minipage}{0.5\textwidth}
            \centering
            \includegraphics[width=1.0\linewidth]{figures/ch5/Mean/CHINA_mean.png}
            \caption{Mean - China}\label{Fig:CHINA_mean}
        \end{minipage}\hfill
        \begin{minipage}{0.5\textwidth}
            \centering
            \includegraphics[width=1.0\linewidth]{figures/ch5/Mean/INDIA_mean.png}
            \caption{Mean - India}\label{Fig:INDIA_mean}
        \end{minipage}
    \end{figure}
    
     \begin{figure}[H]
            \centering
            \includegraphics[width=0.5\linewidth]{figures/ch5/Mean/US_mean.png}
            \caption{\label{fig:US_mean}Mean - United States}
    \end{figure}

    \clearpage
    \newpage

    \item \textbf{Standard Deviation NDVI \& Anomaly distribution over the years}

    \begin{figure}[!htb]
        \begin{minipage}{0.5\textwidth}
            \centering
            \includegraphics[width=1.0\linewidth]{figures/ch5/StandardDeviation/AUSTRALIA_SD.png}
            \caption{Standard deviation - Australia}\label{Fig:AUSTRALIA_SD}
        \end{minipage}\hfill
        \begin{minipage}{0.5\textwidth}
            \centering
            \includegraphics[width=1.0\linewidth]{figures/ch5/StandardDeviation/BRAZIL_SD.png}
            \caption{Standard deviation - Brazil}\label{Fig:BRAZIL_SD}
        \end{minipage}
    \end{figure}
    
     \begin{figure}[!htb]
        \begin{minipage}{0.5\textwidth}
            \centering
            \includegraphics[width=1.0\linewidth]{figures/ch5/StandardDeviation/CHINA_SD.png}
            \caption{Standard deviation - China}\label{Fig:CHINA_SD}
        \end{minipage}\hfill
        \begin{minipage}{0.5\textwidth}
            \centering
            \includegraphics[width=1.0\linewidth]{figures/ch5/StandardDeviation/INDIA_SD.png}
            \caption{Standard deviation - India}\label{Fig:INDIA_SD}
        \end{minipage}
    \end{figure}
    
     \begin{figure}[H]
            \centering
            \includegraphics[width=0.5\linewidth]{figures/ch5/StandardDeviation/US_SD.png}
            \caption{\label{fig:US_SD}Standard deviation - United States}
    \end{figure}
    
    \clearpage
    \newpage

    
    \item \textbf{Histogram NDVI \& Anomaly distribution over the years}

    \begin{figure}[!htb]
        \begin{minipage}{0.5\textwidth}
            \centering
            \includegraphics[width=1.0\linewidth]{figures/ch5/Histograms/AUSTRALIA_histogram.png}
            \caption{Histogram - Australia}\label{Fig:AUSTRALIA_histogram}
        \end{minipage}\hfill
        \begin{minipage}{0.5\textwidth}
            \centering
            \includegraphics[width=1.0\linewidth]{figures/ch5/Histograms/BRAZIL_histogram.png}
            \caption{Histogram - Brazil}\label{Fig:BRAZIL_histogram}
        \end{minipage}
    \end{figure}
    
     \begin{figure}[!htb]
        \begin{minipage}{0.5\textwidth}
            \centering
            \includegraphics[width=1.0\linewidth]{figures/ch5/Histograms/CHINA_histogram.png}
            \caption{Histogram - China}\label{Fig:CHINA_histogram}
        \end{minipage}\hfill
        \begin{minipage}{0.5\textwidth}
            \centering
            \includegraphics[width=1.0\linewidth]{figures/ch5/Histograms/INDIA_histogram.png}
            \caption{Histogram - India}\label{Fig:INDIA_histogram}
        \end{minipage}
    \end{figure}
    
     \begin{figure}[H]
            \centering
            \includegraphics[width=0.5\linewidth]{figures/ch5/Histograms/US_histogram.png}
            \caption{\label{fig:US_histogram} Histogram - United States}
    \end{figure}
\newpage

\end{itemize}




\chapter{CONCLUSIONS AND DISCUSSION}
\label{chap:conclusion}

\section{Conclusion}

The main issue addressed in this thesis is the issue of big data visualization in a mobile application with a focus on user friendliness interface. The tools in this app are designed to serve many comunities. Currently, the mobile application, EyesOnCrops, supports Swift 4.1 language and Xcode 9.2. EyesOnCrops steps for data acquisition method:
\begin{enumerate}
  \item open the app and log in. 
  \item select year and date or any filter options relevance to their needs.
  \item visualize/select their region of interest.
  \item download data so that they can use.
\end{enumerate}

At this time users can make use of these features to visualize the NDVI / Anomaly mean data with the analysis methods explained in chapter 5. 
This chapter concludes with recommendations for major two
categories of further improvement: Data generic mobile app and analysis inside the app.

\section{Future Scope}

An extra thought is to incorporate a discussion forum on the mobile application like a open chat box, similar to  \url{https://www.crowdanalytix.com/} rivalries, where each dataset has its dialog discussion with concentration in Artificial Intelligence, where clients post their code, figures and results in note pads. These permit outside clients to be part of the discussions. Some startup companies look for resources, code, instructional exercises or discourse. For instance, this paper itroduced NDVI Data API in Python, yet clients capable in other programming languages can compose APIs in different languages, like, R, Perl, Matlab, Java or PHP. Questions, can be asked, objections can be posted and be settled/replied by either different clients or the engineers themselves. \\

Secondly, other future aspect would be to make this app completely generic by letting people pick the dataset they want to visualize. For example, there should be a drop down menu of datasets coming from servers, filters, exporting data formats, and databases. Architecture should be generic as well so that it can hold any dataset and ler users visualize their data in the app. \\

Lastly, another enhancement which can be done in future is to have analysis mechanism in the app. For example, if user wants to analyze a specific country for last 10 years, there should be a mechanism to see the mean, standard deviation on the graph in the app itself. 

% 
% The bibliography page, must be between main body and appendices
% 
% You must have thbib.bib file in the current directory 
% 
\bibliographystyle{siammod}
\bibliography{thbib}
% This includes append.tex
\appendix
%
% If you only have one appendix, you should change the above to:
%\appendix
%

\chapter{PYTHON PSUEDO CODE TO GET THE DATA FROM NASA's SERVER AND STORE IT TO PostgreSQL DATABASE}\label{append:python_script_appendix}

\chapter{Python API}\label{append:python_api}

The Python script functionality is provided below which explains how to fetch \gls{ndvi} and anomnaly value from a given point. 

\lstinputlisting[language=Python, breaklines = true, numbers = left]{codes/ndvi_mobile.py}

\newpage

\chapter{PostgreSQL open connection to server function in PHP}\label{append:php_webservice}

\lstinputlisting[language=PHP, breaklines = true, numbers = left]{codes/dbcon_overleaf.php}

\newpage



% (FORMAT) - VERY SPECIAL CASE...  You probably want to delete this!!!
% 
% This shows how you can add special entries to the table of
% contents that DO NOT APPEAR in the document...
% 
% (Yeah, it's not pretty...)
% 
\addtocontents{toc}
{\protect\vspace{\nspextrabaseline}\protect\vspace{-3pt}%
  \protect\hspace{-0.25in}%
  \protect\parbox{0.25in}{Z}SOURCE CODE\space%
  .\protect\hspace{1pt}.\protect\hspace{1pt}.\protect\hspace{1pt}%
  .\protect\hspace{1pt}.\protect\hspace{1pt}.\protect\hspace{1pt}%
  .\protect\hspace{1pt}.\protect\hspace{1pt}.\protect\hspace{1pt}%
  .\protect\hspace{1pt}.\protect\hspace{1pt}.\protect\hspace{1pt}%
  .\protect\hspace{1pt}.\protect\hspace{1pt}.\protect\hspace{1pt}%
  .\protect\hspace{1pt}.\protect\hspace{1pt}.\protect\hspace{1pt}%
  .\protect\hspace{1pt}.\protect\hspace{1pt}.\protect\hspace{1pt}%
  .\protect\hspace{1pt}.\protect\hspace{1pt}.\protect\hspace{1pt}%
  .\protect\hspace{1pt}.\protect\hspace{1pt}.\protect\hspace{1pt}%
  .\protect\hspace{1pt}.\protect\hspace{1pt}.\protect\hspace{1pt}%
  .\protect\hspace{1pt}.\protect\hspace{1pt}.\protect\hspace{1pt}%
  .\protect\hspace{1pt}.\protect\hspace{1pt}.\protect\hspace{1pt}%
  .\protect\hspace{1pt}.\protect\hspace{1pt}.\protect\hspace{1pt}%
  .\protect\hspace{1pt}.\protect\hspace{1pt}.\protect\hspace{1pt}%
  .\protect\hspace{1pt}.\protect\hspace{1pt}.\protect\hspace{1pt}%
  .\protect\hspace{1pt}.\protect\hspace{1pt}.\protect\hspace{1pt}%
  .\protect\hspace{1pt}.\protect\hspace{1pt}.\protect\hspace{1pt}%
  .\protect\hspace{1pt}.\protect\hspace{1pt}.\protect\hspace{1pt}%
  .\protect\hspace{1pt}.\protect\hspace{1pt}.\protect\hspace{1pt}%
  .\protect\hspace{1pt}.\protect\hspace{1pt}.\protect\hspace{1pt}%
  .\protect\hspace{1pt}.\protect\hspace{1pt}.\protect\hspace{1pt}%
  .\protect\hspace{1pt}.\protect\hspace{1pt}.\protect\hspace{1pt}%
  .\protect\hspace{1pt}.\protect\hspace{1pt}.\protect\hspace{1pt}%
  .\protect\hspace{1pt}.\protect\hspace{1pt}.\protect\hspace{1pt}%
  .\protect\hspace{3pt}%
  \protect\space\protect\hbox{on CD}
}


%%%(2013-01-24) "Since the library went electronic, they no longer
%%%require an extra abstract to be inserted at the end of the thesis."
%% 
%% Make the library abstract page
%% 
%\begin{libraryabstract}
%  % This just inserts the the abstract.tex file
%  % You insert your abstract in the space below.

This thesis develops a new iOS app named EyesOnCrops to visualize NASA's Normalized Difference Vegetation Index (NDVI) data in a user-friendly way. NDVI data, in the range between -1 and 1, represent the Earth’s surface vegetation by measuring the difference between near-infrared (for which vegetation strongly reflects) and red light (for which vegetation absorbs). A large positive NDVI value means good vegetation and hence is colored green. NDVI is a huge dataset of over 10 TB, with spatial resolution in ecodistrict and time resolution in 8 days. EyesOnCrops is the first technology that can conveniently and instantly deliver the NDVI data in both color maps and digital data. A user can use an iPhone, an iPOD, or an iPad to visualize the NDVI data at the spatial resolution levels of countries, states, or ecodistricts. Our tool will be extremely valuable to farmers, hedge fund traders and insurance business in addition to teachers and scientists. The thesis includes both the development theory and a user manual of EyesOnCrops. The first chapter reviews the available smartphones technology for spatial data visualization and gives brief introduction about app development methods. The second chapter describes NDVI applications and their significance and explains the EyesOnCrops user manual. EyesOnCrops has an important function of exporting data in the CSV format via email. The third chapter describes the toolkit - NDVI Dataset and its structure, Database Schema, Wireframes of the application and other software packages used for the creation of EyesOnCrops. The fourth chapter describes the process of using an app development toolkit to get NDVI data and converting the data into the format customizable for our mobile application. The fifth chapter made some statistical analysis of the NDVI data. In the end, this thesis also explains how the EyesOnCrops technology can be portable to the applications in other fields.



%\end{libraryabstract}

\end{document}